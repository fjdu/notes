\documentclass{article}

\usepackage{ctex}
\usepackage[a5paper,top=0.8in,bottom=1in]{geometry}

\usepackage{amssymb}
\usepackage[version=4]{mhchem}
\usepackage{hyperref}

\usepackage{setspace}
\setstretch{1.23}

\usepackage{xcolor}
\definecolor{myblue}{rgb}{0.172,0.439,0.729}

\usepackage{marginnote}
\newcommand{\mnote}[1]{\marginnote{\footnotesize\textit{\textcolor{gray}{#1}}}[0ex]}
\newcommand{\fnote}[1]{\footnote{#1}}

\usepackage{natbib}
\setlength{\bibsep}{0.0pt}
\usepackage{../white_paper/astroabrev}
\bibpunct{(}{)}{;}{a}{}{,}

\usepackage{graphicx}
\renewcommand{\figurename}{图}
\newcommand{\reffig}[1]{图~\ref{#1}}
\newcommand{\refstep}[1]{第~\ref{#1}~步}

\newcommand{\mitem}{$-$~}

\usepackage[inline]{enumitem}

\usepackage{xeCJK}
\setCJKmainfont[BoldFont=STHeiti, ItalicFont=Kaiti SC Bold]{STSong}
\setCJKsansfont[BoldFont=STHeiti]{STXihei}
\setCJKmonofont{STFangsong}

\usepackage{fontspec}
\setmainfont[Ligatures=TeX]{Georgia}
\setsansfont[Ligatures=TeX]{Arial}

%\usepackage{titlesec}
%\titleformat{\section}[block]{\Large}{\chinese{section}、}{1ex}{}
%\titleformat{\subsection}[block]{\large}{\arabic{subsection}.}{1ex}{}
%\titleformat{\subsubsection}[block]{}{(\arabic{subsubsection})}{1ex}{}

\newcommand\refeq[1]{(\ref{#1})}
\newcommand\from[2]{\subsection{{#1} {#2}}}
\newcommand\said[1]{#1}

\title{60 米级亚毫米波望远镜 \\ {“宇宙物质与生命起源”工作组} \\ {讨论记录}}
\author{
\begin{tabular}{ll}
\hline
\multicolumn{2}{c}{工作组成员 (按拼音顺序排列)} \\
%
常强 & \url{changqiang@xao.ac.cn} \\
杜福君 & \url{fjdu@pmo.ac.cn} \\
李娟 & \url{lijuan@shao.ac.cn} \\
李小虎 & \url{xiaohu@nao.cas.cn} \\
秦胜利 & \url{qin@ynu.edu.cn} \\
全冬晖 & \url{quandh@xao.ac.cn} \\
杨涛 & \url{tyang@lps.ecnu.edu.cn} \\
张泳 & \url{zhangyong5@mail.sysu.edu.cn} \\
\hline
\multicolumn{2}{c}{记录人:杜福君} \\
\hline
\end{tabular}
}

\begin{document}

\maketitle

\section{讨论记录}

\from{
Xiaohu Li <xiaohu@nao.cas.cn>
}{
Jul 28, 2018, 11:34 PM
}
\said{
很高兴加入这个项目,我考虑一下,加一些evolved star的东西。

祝好,

小虎
}

\from{
杨涛 <tyang@lps.ecnu.edu.cn>
}{
Jul 28, 2018, 11:33 PM
}
\said{
各位老师好,
我是华东师范大学杨涛,2017年从夏威夷大学博士后回国,目前在精密光谱科学与技术国家重点实验室从事物理精密光谱与精密测量(电子电偶极矩测量),以及分子动力学方面的天体化学实验工作。近日参加星际物理与化学会议,感受颇多,现与各位老师分享下我在天体化学方面想要开展的工作,希望能够抛砖引玉。有些专有名词翻译不到位,请包涵。

最近在与夏威夷大学Ralf I. Kaiser教授共建一个项目,主旨是在超高真空表面散射装置(Ultra-high Va­cu­um Surface Scat­tering Ma­chi­ne)中引入由甲烷、氨气、水蒸汽等简单的小分子气体,使用液氦脉管制冷方法在低温表面模拟类冰物质生成,之后通过电离辐射(离子源、电子源轰击)或VUV光源作用于类冰物质(模拟天体环境中的辐射条件),再使用程序升温脱附(Temperature Programmed Desorption),当物质升华为气态时使用可调无碎片真空紫外光电离方法(基于同步辐射光源)探测生成的各类分子。由于使用同步辐射光源进行单光子电离探测,且配合灵敏的反射飞行时间质谱,因此可以得到质谱、光电离截面 (Photoionization Cross-section)曲线(同分异构分子的光电离截面曲线迥异)、各升华温度下的分子种类等相关信息,而对这些多维信息的解析可以获得某一辐射剂量、某一小分子配比情况下产生的产物分子的种类以及占比。通过寻求理论计算与天文学建模方面的帮助,我们相信实验结果或许能从物理化学反应层面揭示小分子到小有机分子的天体化学演化路径。

项目由华东师范大学与夏威夷大学共建,由中国科学技术大学同步辐射国家重点实验室提供光源,华东师范大学杨涛、夏威夷大学Ralf I. Kaiser、中科院化学所郑卫军负责实验,中国科学技术大学潘洋与杨玖重负责协助搭建实验平台,佛罗里达国际大学Alexander M. Mebel、中国科学技术大学张凤负责计算。我们期待我们的工作能够加入到星际物理与化学的大家庭,若项目进展顺利,明年就会相关实验结果。
}

\from{
全冬晖 <quandh@xao.ac.cn>
}{
Sat, Jul 28, 10:29 PM
}
\said{
福君,小虎,杨涛几位,

我的一点浅见如下:

1.1.2, 热分子云核
生成有机大分子的过程,建议加上另一种可能,就是低温下聚集在宇宙尘埃的小分子可以在尘埃表面反应(与扩散进行速率竞争),等到温度升高或者适合的条件,比如激波,再返回到气相中。

另外,Sgr B2也是一个很好的有机大分子甚至生命前分子的源,其中已被观测到的生命前分子比如CH3CHNH (Loomis et al. 2013), HNCNH (McGuire et al. 2012), HNCHCN (Zaleski et al. 2013)等。

1.2 搜寻生命前分子和生命分子
考虑加上另一大类的生命前分子,也就是含氮有机分子,很容易形成氨基酸。比如(Elsila et al. 2007)
\ce{CH3CHNH + UV + HCN -> NCCH(CH3)NH2}

\ce{NCCH(CH3)NH2 +H2O -> HOOCCH(CH3)NH2}

目前就想到这么多,回头我再详细参详。
}

\from{
Jinhua He <jinhuahe@ynao.ac.cn>
}{
Jul 28, 2018, 5:22 AM
}
\said{
谢谢福君的详细解释,我终于明白了你的巡天的原意。但是,对起源演化的研究,我之前想象的是要有针对性地对选定的分子谱线加连续谱在一大批选定的天体中进行巡天观测,而且还会希望有较高的空间分辨率,尽量分解出天体中的化学分布结构,因此不一定限于低于260GHz的较长波段。所以我之前想象的一直都是两个在观测操作和目标设计上都独立但在具体研究上有一定重叠的巡天项目。我想象的无偏巡天的科学目标就是简单地去看:银河系中有机物都在哪里?一共有多少?足不足以供给地球和其它星球上的生命形成?供你们参考吧。

你当前设计的巡天,不但有机分子看了,许多无机分子也看了。这样一来,仅仅是生命起源这个科学目标,似乎就不太足以支撑起这样的宽波段巡天了,至少得把银河系结构性质和分子云恒星形成等研究加进去才够。不知道在巡天方面,你们谁跟恒星形成团队和银河系结构研究团队有过交流协同吗?

如果不是因为谱线干扰问题,ALMA在探测来自致密的热云核的氨基酸等复杂分子方面肯定是很有优势的:空间分辨率和灵敏度高。而如果确实存在低温的氨基酸分子的话,那么SKA是具有绝对优势的。我们这个60米,只有在存在空间上较为弥散(比如10角秒尺度,波束稀释效应不大),温度中等(最强辐射出现在我们的观测波段)时,才可能说是对SKA和ALMA有优势,但跟LMT50m比起来又没有优势可言了,我们的好主意都会被LMT先一步实现。而且如何说明太阳附近确实存在这样的甘氨酸发射区呢?如果我们不能保证第一个探测到,这样的探测项目就没多少科学意义了。可以说,但也只是说说而已。也许可以考虑其它重要的糖脂核酸分子的首次探测?(但也逃不出LMT在前的紧箍咒,呵呵。)
祝好!

金华
}

\from{
Jinhua He <jinhuahe@ynao.ac.cn>
}{
Jul 28, 2018, 4:42 AM
}
\said{
大家好!

谢谢福君的更新,特别是在你们都在很忙地参加会议的时候。

我主要看了一下计算公式1.1.按照目前给出的参数,完成北天巡天的观测时间确实在1年量级。只需要这么短的时间就能完成当然是个很好的事情。另外,能不能也提及南天?我们为什么不做全银道面巡天,而仅仅满足于做中国本土能见的北天呢?以前的CO 1-0全银道面巡天就是个很好的先例。南天也有很多有趣的天体,如银心,eta carina,大小麦哲伦云,不少有趣的近邻星系等等。

我们是不是也应该适当提及设置这些巡天参数的依据?比如说最重要的Tsys和rms,为什么是100K和10mK呢?前者与仪器和台址有关,后者与科学目标有关。从经验角度看,Tsys=100K也许问题不大,但给个基本的说明也许更好。作为一个巡天项目本身,说rms=10mK超过了之前的巡天,这也问题不大。但是从科学的角度,10mK能如何解决我们要解决的关键科学问题呢?我觉得这个才是我们写科学目标时需要重点关注的问题,而目前的版本中还没有提及太多。我自己的调研中也还没有来得及整理这类定量信息,还需要更多努力。

对这个问题,我们也可以换个角度来看。我们做巡天观测,涉及到参数空间。除了二维空间位置外,还有频率覆盖和灵敏度参数(也隐含涉及到频率和空间分辨率),至少四维参数空间吧。我们需要说明我们想要观测研究的对象或者现象确实会在我们的巡天参数空间范围内大量出现,才能证明研究可行。如果这些天体现象还仅仅出现在60米独特的参数空间内,那就是我们的独特项目,值得大声宣称了。没有这样的考虑,就等于我们仅仅依赖技术进步,盲目去巡天,巡完了造一堆数据,再去看能做点儿什么科学。这是过去缺少科学目标的项目模式,现在仅仅做到这个程度应该是不够的了。

对于后面其它的科学目标部分,我大概浏览了一下。我的感觉也是我们的科学目标还不够明确,主要提到了别人做了什么。但我们自己要做什么呢?如何做得到呢?这是我们科学团队需要思考并阐明的问题。你们觉得呢?

另外,对于甘氨酸的谱线的计算,建议给出计算的文献信息,以表明我们计算的可靠性。我建议你们也参考一下Jimenez-Serra et al. (2014ApJ…787L..33J),她也计算了类太阳恒星形成区hot corino L1544和大质量hot core AFGL2591中甘氨酸在我们感兴趣波段的谱线辐射强度。她预测谱线峰值强度在L1544中达到了10mK,但是她假设了很高的甘氨酸丰度$10^{-8}$。另外,探测到一条甘氨酸的众多谱线并不足以证认它,特别是在1mm,其它谱线干扰是个重要问题。比如Jimenez-Serra et al.提到对hot core而言,干扰问题很严重。对60米而言,可能还需要考虑波束稀释效应,除非你们能证明甘氨酸的分布真的能在某个天体中延展到10角秒以上。从Jimenez-Serra et al.的论文发表到现在,应该说还是没有可信的甘氨酸探测发生。总之,探测到甘氨酸是很不容易的,需要小心论证。

祝好!

金华
}

\from{
Jinhua He <jinhuahe@ynao.ac.cn>
}{
Jul 27, 2018, 10:34 AM}
\said{
谢谢福君列出框架草稿,我们有了一个很好的开头。我有如下几点供大家讨论:

1)虽然60米可能很希望首次探测到甘氨酸,但是最基本的生命分子并不只氨基酸,还包括脂肪、核苷酸和碳水化合物。参见我之前发给你们的Sect.3.3.5 of the Astrobiology Primer v.2.依照这样生命分子的分类去写会比较清楚些。首次探测另外三类中的某些分子也许也有趣?

2)说“未来的⼈类若要持续发展,必将前往太空”,这可能不一定哦。这个是以我们无法保护好地球,像《星际穿越》电影里讲的那样必须逃离地球为前提的。但我们也可以保护好地球,做到在地球上可持续发展呀。这样说会不会容易引起争议呢?

3)后面的1.2节及以后所有章节都是60米可做的,但又都不是60米的优势项目。比如说我们最希望60米拥有的对有机物甘氨酸等的首次探测这个目标,60米都是无法与类似SKA和ALMA这样的干涉仪竞争的,无论是波段、灵敏度还是空间分辨率都没有优势。我们何不改变思维方向,让60米与SKA和ALMA配合、共鸣,去专注地做介于SKA和ALMA两者之间的或者与它们互补的重要科学呢?我看到国际上大望远镜项目都会提到与其它大设备的synergy这样的说法。所以,对于这些不是优势的方面,我们可以说它们,这样可以让读者感觉我们写得比较全面。但是显然也应该在某个显眼的地方给60米的优势项目留下足够的章节空间。有机物(或者也包括其它无机分子?)无偏巡天是一个优势项目,另一个优势项目是通过对特定类别的天体的系统性巡天观测,来研究有机分子的起源演化问题。而这后者在目前的框架草稿内还无处安放。

如果你愿意单独开辟这个章节的话,我可以建议一个类似这样的题目:星际有机物的起源演化。这与无偏巡天并不重叠,无偏巡天仅仅是问大尺度分布如何,而起源演化问题则需要更深入的物理、化学探索。无偏巡天的结果对起源演化的研究有帮助,但是仅靠它是不够的。对选定天体的更详细的观测研究是必不可少的。你们同意吗?

也许你是想把这个与搜寻新分子的目标一起放在1.2节吧?首先,我觉得搜寻新分子本身是一个单独的话题。而且这个话题的性质与无偏巡天有点儿类似,就是探测问题。我感觉并不适合将它与起源演化这样复杂的问题放到一起去阐述。因此,我目前将1.2节理解为其重点是探测新分子或新现象(如手征性)。

我说得不一定对,欢迎表达不同的看法。

祝好!
金华
}

\from{
Jinhua He <jinhuahe@ynao.ac.cn>
}{
Jul 17, 2018, 11:21 PM
}
\said{
谢谢福君回复。

广域分布和起源演化这两个方面,我也觉得不可能是截然分开的,只是侧重点不同而已,可以相互参考。等写得差不多了我们再具体讨论。
我说的原恒星盘与原行星盘应该是一个对象,等统稿时可以统一一下术语。

那我就再领取热云核吧,等有了结果再向各位汇报。

祝好!
金华
}

\from{
Fujun Du <fjdu@pmo.ac.cn>
}{
Jul 17, 2018, 10:50 PM
}
\said{
谢谢金华写的关于PDR的内容!我觉得已经很好了。

关于“生命前分子” (prebiotic molecule),我的理解是它指的是已经具有相当的复杂程度、与生命相关的分子,所以我觉得关于PDR化学的内容似乎比较适合2.3.1节,即“星际有机物的广域分布”,并且它们的起源应该不局限于恒星晚期和超新星遗迹。不过关于这些划分我也不确定。

PDF文件里的原行星盘和邮件里的“原恒星盘”应该是同一个东西吧?即protoplanetary disk。

热云核和冷云核你可以继续写吧,应该没有什么冲突。
}

\from{
Jinhua He <jinhuahe@ynao.ac.cn>
}
{
Jul 17, 2018, 1:01 AM
}
\said{
大家好!

抱歉我的第一个调研报告来得比我预期的晚了点儿。我写的HII PDR部分见附件。为了避免格式兼容问题,我先分享PDF供各位评阅。

我对HII PDR的调研显然很不完整,也没有花时间去关注模拟研究的进展情况,但是还是有了些眉目,得到些新思路。我感觉原来所提的一些天体类别(如弥散云、半透明云translucent clouds、HII区、原恒星盘)以及巨分子云边界面都可以合并为PDR一类,因为它们很可能都主要是由于PDR FUV光子作用于冰物质而产生复杂有机分子并且被解吸附到气相被观测到的。PDR的广泛存在也可以成为我们开展无偏巡天观测的理由之一。这样一来,我们原来的2.3.2节的提纲也许可以更新为这样:

2.3.2 探测生命前物质
(说明:提出若干目标分子、若干目标源,估算需要多少时间能探测到多强的信号,能为认识生命的星际起源提供哪些帮助。)

2.3.2.1)生命前物质的起源
2.3.2.1.1)演化晚期恒星(我建议调研之初不要仅限于PNe)
2.3.2.1.2)SNRsx

2.3.2.2)生命前物质的传播、演化路径和主要场所问题
2.3.2.2.1)PDR(弥散云、半透明云translucent clouds、巨分子云边界面、HII区,原恒星盘)
2.3.2.2.2)冷云核
2.3.2.2.3)热云核
2.3.2.2.4)喷流区
可以将原恒星盘提出来到后面单独强调其重要性。如果时间不够,喷流区可以删除,因为我感觉喷流区可能不会产生多少新的有机分子,只是能够解吸附一些。对于PDR,我目前主要关注了HII一类,其它类别可以等后面有时间了再去增补;如果时间不够,也可以在第一稿中仅仅泛泛地提一下,等以后改进白皮书第二版的时候再说。

如果没有其他人在做热云核的话,我打算再扎个猛子去热云核探探情况。否则,我也可以做冷云核。

欢迎大家给我的HII PDR调研材料以及上面的看法多提批评意见,我将尽力改进。谢谢!

祝好!

金华
}

\from{
Jinhua He <jinhuahe@ynao.ac.cn>
}{
Jul 13, 2018, 3:07 AM
}
\said{
新的强劲的竞争者来了!:
AtLAST,ALMA的50米亚毫米波望远镜项目。
}

\from{
Jinhua He <jinhuahe@ynao.ac.cn>
}{
Jul 9, 2018, 9:48 PM
}
\said{
福君:

谢谢进一步解释!那么2.3.2节主要就是星际有机物的起源演化问题了。那我建议将2.3.2节根据前面的讨论再细分,才方便我们各自领取不同的小方向去开始调研。为了加速讨论,我这里就越俎代庖提一个2.3.2节的进一步细分建议供参考了哈(其实就是之前的版本):

2.3.2)探测生命前物质

2.3.2.1)生命前物质的起源问题
2.3.2.1.1)演化晚期恒星(我建议调研之初不要仅限于PNe)
2.3.2.1.2)SNRs

2.3.2.2)生命前物质的传播、演化路径和主要场所问题

2.3.2.2.1)弥散云和半透明云translucent clouds
2.3.2.2.2)分子云
2.3.2.2.3)冷云核
2.3.2.2.4)热云核
2.3.2.2.5)喷流区
2.3.2.2.6)HII区PDRs

这样行吗?请明确认可或者提出替代提纲,我们才好开始动手。

另:是不是可以先做个浅调研,主要凭经验和少数文献先写个文本,吃个定心丸,后面再根据具体情况逐步讨论改进?

金华
}

\from{
Fujun Du <fjdu@pmo.ac.cn>
}{
Jul 9, 2018, 3:23 PM
}
\said{
参考之前的讨论,再把2.3节细化一下:

2.3 使用60米望远镜探测有机分子、生命前分子、生命分子的前景
2.3.1 探测星际有机物的广域分布
根据现有知识,提出巡天的深度、空间范围、频率覆盖范围、频谱分辨率,罗列能被囊括进来的重要分子。
2.3.2 探测生命前物质
提出若干目标分子、若干目标源,估算需要多少时间能探测到多强的信号,能为认识生命的星际起源提供哪些帮助。
2.3.3 探测生命物质
类似于2.3.2
}

\from{
Fujun Du <fjdu@pmo.ac.cn>
}{
Jul 9, 2018, 1:59 PM
}
\said{
参考何金华的建议,我这里重发一遍之前的提纲,并对第二部分增加了一些描述,有看法请继续提出:

1. 引言
1.1 毫米波分子天文学简史
1.2 往亚毫米波段扩展的必要性

2. 星际有机物和生命分子
2.1 复杂有机分子的发现
这一节是综述,讲已发现的各种分子,特别是有机分子。会讲观测手段、观测的源的特性,等等。
2.2 复杂有机分子与生命分子的关联
这一节带有推测性,把话题从分子天文导向天体生物 (astrobiology),讲天文观测、特别是对复杂有机分子的观测对认识生命起源的意义。
2.3 使用60米望远镜探测有机分子、生命前分子、生命分子的前景
这一节比较具体和实在,从技术细节的角度讲60米望远镜能在探测有机分子、生命前分子、生命分子方面做什么,会讨论不同的观测策略 (不同的科学目标有不同的观测策略)、预期观测的源的类型和/或覆盖的天区、达到什么样的观测深度、得到什么样的结果。这里并没有预设只观测特定类型的源 (比如Sgr B2或者IRAS16293那样的),完全可以把行星状星云、超新星遗迹等类型的源包含进来,也没有预设只做无偏巡天。
这一节篇幅可能会显著长于其它节,有必要的话可以再拆分,写出来了再说,现在还没写。
2.4 星际分子到行星系统的迁移
这一节讲物质从星际空间转移到行星系统的过程,其实主要就是讲原行星盘,包括在原行星盘里探测复杂分子的可能性,或者就算不能 —— 对原行星盘的研究能对厘清“星际物质-星周物质-生命起源”这个逻辑链条做出哪些贡献,以及60米望远镜能做些什么 (比如通过巡天能力提高样本数以供后续研究)。
2.5 星际物质与太阳系内天体物质构成的关系
接上一节,太阳系 (our solar system) 这样的行星系统的物质终究是来自原行星盘的气体和尘埃,那么对太阳系内天体的研究可以为“星际物质-星周物质-生命起源”这个链条提供更多佐证;对其它行星的观测也可以为认识地球环境的特殊性提供对比。这里会讨论60米望远镜能不能对太阳系内天体的观测做出贡献 (如果不能,就删掉这节)。
}

\from{
Jinhua He <jinhuahe@ynao.ac.cn>
}{
Jul 9, 2018, 11:41 AM
}
\said{
那很好。看来我目前还不太理解的部分主要是:第2.3节里有机分子巡天的内容到底指的是什么?我对这部分原来的理解仅仅是选择某些对生命分子的合成最重要的有机分子进行全银道面频谱成图巡天观测,给我们指出有机分子在银河系的哪里出现的大图景。这里面存在的关键问题是选哪些分子的哪些谱线巡测?(根据胜利之前的邮件似乎他认为是要做60米的全波段巡测,是吗?)为什么是mK量级的巡天深度?巡天的意义何在?为什么要全银道面或全天巡天,而不是重点选定区域巡天?我个人觉得至少要把这些问题说清楚,才能端出一个令人信服的巡天计划来作为望远镜立项对驱动项目。

我在前面邮件里提到的有机分子的起源演化问题,我个人的感觉是不能等同于上面我所描述的有机分子巡天项目的,虽然后者也确实能提供些信息帮助我们理解起源演化问题。这是因为我们在这部分可能需要对关键天体环境进行更加详尽的观测研究,并且最终将各个环境下有机分子的起源演化状况串联起来,才能得到一个银河系内“元素--分子--有机分子--生命分子--地球生命”的完整图像。我觉得60米应该有能力去获得这个完整图像。所以,我目前对这部分的设想是对每一类天体应该都有一个规模较大,能体现60米优势的巡天观测,它们被“有机分子起源演化”这条线索串起来形成一个整体。(欢迎提供不同的设想!)而这部分在福君解释的提纲里似乎是没有的。如果你们认为这部分可以这样存在的话,那么2.4和2.5其实就是这个有机分子起源演化问题的内容之一了。当然了,为了强调近邻天体的观测与生命起源问题更加贴近的关系,我不反对将它们单列出来,作为次重点或者附加的可做部分出现。但我至少没觉得有机分子起源演化问题是与2.3节重叠的内容。能把它作为新的2.4节吗?

金华
}

\from{
Fujun Du <fjdu@pmo.ac.cn>
}{
Jul 9, 2018, 9:43 AM
}
\said{
2.4节是原行星盘,2.5节是太阳系内天体,包括行星、卫星大气成分及其可能的星际起源。

2.1节讲天文,属于review段落,2.2节讲生物及其与天文的关系。

2.3节是重点,讲有机分子巡天,2.4节是次重点,2.5节按我现在的理解不是重点,只是逻辑顺序里的自然的下一步。
}

\from{
Jinhua He <jinhuahe@ynao.ac.cn>
}{
Jul 9, 2018, 9:29 AM
}
\said{
福君:

我之前说的细分只是2.4节内容的细分。经你们提醒,我感觉第二章的各节内容的分野可能还需要说明得再清楚点儿,才方便操作。这里我拷贝你之前发的第二章内容

2. 星际有机物和生命分子
2.1 复杂有机分子的发现
2.2 复杂有机分子与生命分子的关联
2.3 使用60米望远镜探测生命前分子和生命分子的前景
2.4 星际分子到行星系统的迁移
2.5 星际物质与太阳系内天体物质构成的关系
我目前的理解如下,如果我错了,请务必悉数指出:

第2.1与2.2节内容应该不是60米的研究主题,属于常识介绍性的内容,是本章的引入段落。

第2.3和2.4节内容(稍稍完善之后)应该包括我们更早之前邮件中提到过的60米的两个主要潜在的创造性研究方向或课题:有机分子无偏巡天;有机分子(包括生命分子)的起源演化。这个应该是重点阐述的部分。

第2.5节内容与原行星盘观测方向类似,可以成为60米的研究内容,但并非60米的独特项目。(也许原恒星盘和太阳系天体的观测可以合起来当作0.5个重点来对待?)

这样一来,在这个项目中我目前就看到这2.5个重点课题方向,是我们值得花力气去写好的。

我之前将2.3节理解为了无偏巡天,将2.4节理解为了有机分子的起源演化。看来这并非你的原意。请福君评论一下我上述理解你是否同意?希望你再详细一点儿地解释一下2.3和2.4节的内容范围的分野。谢谢!

金华
}


\from{
Jinhua He <jinhuahe@ynao.ac.cn>
}{
Jul 7, 2018, 4:21 AM
}
\said{
谢谢福君!我的一点点建议:

引言部分建议不要跟整个项目的总引言内容重叠(他们肯定会写的),我们重点在有机分子和生命诞生方面。

目前这个提纲比较系统全面,是中规中矩的。看起来我们目前的重点应该是在2.3和2.4节了。3.1节或许也有点儿意思,可以与银河系巡天结合得比较好。但是我估计主要是丰度高的简单些的有机分子会被探测到。剩下的可能都是“可做而非独特”的项目,适合于简略提及。你们看在章节安排上要不要适当突出重点呢?还是保持这个结构,仅仅通过章节的篇幅大小来反映重点所在?

如果你们愿意改变章节结构,我建议可以这样:先给出一两段的背景简介,并引出我们的主打项目;然后直接切换到主打项目的叙述;最后给个一两段简单说说也可以做的其它方向(比如2.5,3.2,以及一般天体化学主题等等)。

希望大家多发言啊!特别是如果你们觉得我的建议并不好,欢迎提出你们的建议。我们这个工作组目前来看都还比较沉闷,需要更活跃的工作氛围。
金华
}

\from{
Fujun Du <fjdu@pmo.ac.cn>
}{
Fri, Jul 6, 11:14 AM
}
\said{
大致列了一下提纲,大家看一下:

1. 引言
1.1 毫米波分子天文学简史
1.2 往亚毫米波段扩展的必要性

2. 星际有机物和生命分子
2.1 复杂有机分子的发现
2.2 复杂有机分子与生命分子的关联
2.3 使用60米望远镜探测生命前分子和生命分子的前景
2.4 星际分子到行星系统的迁移
2.5 星际物质与太阳系内天体物质构成的关系

3. 宇宙中的有机物分布
3.1 近邻星系中的有机物分布
3.2 更远星系中的物质组成
}


\from{
Fujun Du <fjdu@pmo.ac.cn>
}{
Jun 28, 2018, 6:02 AM
}
\said{
我个人觉得,在目前的阶段,关于望远镜的技术参数,只能是做数量级上的估计,给出“可以做”或“不可能做” (就像“一年能做完”跟“一百年才能做完”的区别) 的结论。
具体一个巡天项目是一年做完,还是两年、五年做完,在目前的阶段可能没办法精确计算。
一些技术问题,如果没有基本物理规律的限制 (比如量子不确定关系那样的),只要努力大概总是可以克服的,就看能够投入多少资源了。

当然,根据当前的技术能力,以及具体实现上的不确定性和资源的限制,为今后的巡天工作需要的观测时间估计一个范围,的确是有意义的。
但对这些技术不确定性的估计本身也不容易,比如面板精度,可能目标是达到25微米,实际能做到35微米、还是超额完成达到20微米,我觉得不好估计。
一方面,西部高原环境恶劣施工难度大,可能会影响拼装精度,但另一方面,也许未来十年会有什么科技进步能帮助提高精度。所以不太好估计。
我可以再问问技术人员 (各位如果跟做技术的人熟也可以咨询一下),看他们可否根据以往经验,给出“期望达到的技术指标”与“实际达到的技术指标”之间的典型差异值。

300GHz以上的波段确实会比较困难,不过这还是回到老问题:300 GHz 以上 (比如 300 - 500 GHz) 有没有特别重要的科学?
如果有的话,也许也不能轻易抛却,因为有可能这个科学目标是如此的重要以至于可以把大部分观测时间都给它 (这里只是说存在这种可能性,至于真的有没有这样的科学目标还不清楚)。

杜福君
}

\from{
Jinhua He <jinhuahe@ynao.ac.cn>
}{
Jun 20, 2018, 10:17 PM
}
\said{
谢谢福君咨询这些技术问题。我很高兴技术团队在经过深入思考之后,仍然确认1度大视场是可以做到的。我不是技术专家,我很乐意信任他们的判断。为了让我们尽快进入考虑科学目标的状态,我提出如下建议,大家看看如何:

1)虽然1度视场也许可为,但是偏离中心最远的外围波束可能效果比较差些,我们可以假设一个0.5度或者最多0.75度的有效视场大小来设计主导的科学目标,这样的巡天速度仍然可以到达LMT 8角分视场的14-32倍;

2)增大副镜以及添加改正面板等也许会增大error beam和spill over,我们可以预计主波束的效率比一个理想的球面60米口径的主波束效率要低些,或许也可以对主波束效率打个7折来设计科学目标;

3)基于类似的原因,我们的空间分辨率有可能比理想的60米口径望远镜稍差一点,假设分辨尺度放宽20%;

4)由于旁瓣增强,数据处理的压力可能会增大(数据处理速度是否会成为一个瓶颈呢?),成图信噪比可能也会变差一些,需要比理想情况适当延长观测时间,比如假设1.5倍的观测时间。

5)福君得到的回复里面没有涉及到是否能够让60米的大镜面实时保证25um的总体面形精度的问题。即便真的能够勉强做到,我估计那也是达到我们目前技术的极限了吧。在这种极限情况下,望远镜的口面效率比较低,error beam进一步增大,因此这将不是这个望远镜的一个优势性的观测能力。根据之前在微信群大讨论中张智昱(是他吧?)提供的ppt里提到的那样,我们不能为30%以下的观测时间去设计主导科学项目。因此我建议我们以300GHz波段为主要优势波段来设计我们的有机物巡天,这样对西藏台址条件的压力也会小些。LMT虽然声称可以做到300GHz,但无论从望远镜面板精度和台址条件讲,那都是他们的极限了,无法与我们的无偏巡天抗衡。而在更低的3mm波段,我们与LMT比起来的优势就小得多了。

另外,硕大的副镜设计意味着将来观测点源时可能将无法使用beam switch模式(或者他们可以设计一个可振动的第三镜来代替?)。但是这个也许可以用多波束的优势加以弥补。

我建议的这些数值,纯粹是一种直觉。欢迎大家说出你们认为合理的估计数值。建议福君最后将我们达成一致意见的这些估计数值提交给技术团队确认为妥。

祝好!

金华
}


\from{
Fujun Du <fjdu@pmo.ac.cn>
}{
Jun 20, 2018, 11:28 AM
}
\said{
关于大视场的问题,问了一下做望远镜的人:

\begin{quotation}
关于为什么 LMT 没有强调大视场,可能是由于他们科学方面没有这方面的特别的需求,为了降低造价和研制难度就没有做很大。

大视场带来的问题包括:1. 副镜会比较大,造成中心遮挡;2.  旁瓣和交叉极化的水平会增大;3. 可能需要在仪器前端使用改正镜或改正板,从而引入额外的损耗。这些都可以根据科学的具体要求 (对旁瓣和极化纯度的要求) 进行优化设计。

大视场的技术挑战主要在仪器方面:需要大规模的仪器阵列来填充望远镜的视场。不过仪器可以从小到大分批安装,逐步扩充。
\end{quotation}

我觉得前期阶段科学目标的规划可以先按照比较理想的情况来,以后如果技术方面有调整,再跟着调整,就像一个迭代的过程。

}

\from{
Jinhua He <jinhuahe@ynao.ac.cn>
}{
Jun 13, 2018, 1:29 AM
}
\said{
经过这些讨论,现在确实更清楚些了。但是我感觉仍然有些关键点不够清晰,也要留意潜在的陷阱。

我也希望我们在西藏台址会比LMT 4600米的台址好很多。但是我们也得留意西藏海拔高不一定水汽少。虽然卫星资料分析显示有潜在好台址,具体选址下来的结果还是可能会有些变数的。万一没那么理想呢?那我们科学团队就要承担科学项目做不成功的压力了。至少在设计科学目标的时候要论证如果台址差些会不会有导致观测设计失败的风险。
对于1度大视场,那确实是令人我们观测者心仪的一个优势。但是,当我看到50米的LMT只有8角分视场的时候,心里就又点儿打鼓了。如果做大视场容易,那为什么LMT不做呢?做大视场费钱吗?技术难吗?做大视场会不会带来其它方面的缺点呢?比如说,会不会旁瓣很高,或者单个波束增益很差?或者多波束成像质量会变差?等等。我希望技术团队能够确认为什么我们可以有这个优势,而欧美人没有。我们也可以与LMT50m,IRAM30m,Nobeyama45m等的镜面设计比较一下,给出一个大概的但要令人信服的技术可行性说明。我也希望技术团队不但要告诉我们这个望远镜设计有什么优势,也要告诉我们有哪些不足。不足的方面就是坑儿,我们不能踩。这个大视场显然对我们的有机分子巡天很重要,因此值得先把这个基础搞确切了再动手设计具体科学目标。

另一个技术忧虑是我们做60米去观测600GHz,要求镜面整体面形精度达到小于25微米。这个难度可能不小。LMT为什么仅仅勉强做到了0.8mm观测波段?但愿他们要么是为了省钱,要么是因为台址限制,因此才给我们60米留下了一个0.8mm-0.5mm的独特技术参数空间可以发挥。如果LMT是因为台址限制,那我们的台址质量也需要搞确切,才有理由决定去做这么高观测频率的望远镜。无论如何,最好要认真核实一下我们的技术指标确实可以稳妥实现,我们才有信心去畅想高频谱线观测的科学目标。

各位以为如何?

金华
}

\from{
Fujun Du <fjdu@pmo.ac.cn>
}{
Jun 12, 2018, 9:34 AM
}
\said{我能想到的、相对LMT的独特优势就是:台址和视场,因为这两者是“固化”的(至少台址是,视场我不完全确定),而其它后端设备都可以在后期不断升级改造,我觉得目前还很难说有什么仪器是国内能做而国外不能做的。

这两点优势体现在观测上就是观测的速度和深度,也就是在给定时间内能达到多低的噪声水平。根据这个网址 (\url{http://www.lmtgtm.org/general-2/general/}),LMT的视场直径是8角分,这样的话跟之前提及的60米望远镜的1度差别还是相当大的:(1度/8角分)$^2$=56.25。

能不能因为这样大的视场差距而说60米望远镜相当于56个LMT?我不确定,但至少差别是明显的。这样大的差距导致的就是巡天项目时间花费的量级上的巨大差别,这对于“无偏巡天”肯定很重要。台址导致的大气不透明度差异对观测效率也会有类似的效应 (甚至效应更大,会让“不可能做”的事情变成“可以做”)。

当然如金华之前说的,我们需要定义什么“是无偏巡天”。可以有几种理解:
\begin{itemize}
  \item 不局限于特定天区
  \item 不局限于特定分子
  \item 不局限于特定波段
\end{itemize}  

考虑到不同类天体的差异,可能需要参照之前金华说的那样,对不同层次、不同距离的天体分类,不同类的天体采用不同的巡天策略 (空间和频谱分辨率、波段、积分时间、观测模式),而不是“盲巡”,否则可能会造成一些“浪费”。

这种分类既是技术上的(避免时间浪费),也是科学上的(不同类天体关注的重点不同):比如在星系里(包括高红移星系)探测到复杂分子可能就算成功,而在河内的某些源里是希望探测到更多生命前分子乃至生命分子。

参照之前金华给的列表,我这里再略微修改一下可供参考:
\begin{itemize}
  \item 宇宙中的有机物分布 (在星系中找复杂分子)
  \item 星际有机物和生命分子 (银河系内的复杂分子、生命前分子、生命分子)
  \item 星际分子到行星系统的迁移 (原行星盘等)
  \item 近邻天体的分子化学 (太阳系内天体)
\end{itemize}  

杜福君
}


\from{
Jinhua He <jinhuahe@ynao.ac.cn>
}{
Jun 11, 2018, 11:27 PM
}
\said{
LMT声称要做到 0.8 毫米波!而且人家望远镜已经建好了,也有米国人合作,是我们的危险对手。我们如果提个LMT也可以做的巡天项目,分分钟就成了LMT的盘中餐。如果我们仅仅提出一个LMT也可以做的科学目标,这朵花怕是可以准备提前凋谢了。

金华}

\from{Sheng-Li Qin <qin@ynu.edu.cn>}{
Jun 11, 2018, 10:08 AM}
\said{
需要大家讨论,观测那些分子,解决那些问题。

对同一个分子观测尽可能多,才能得到准确的物理和化学参量。

需要我们对技术设计提要求,带宽越大,能同时得到的谱线越多,信息量越大。SMA目前在灵敏度、分辨率,uv覆盖上没有优势了,就把带宽做的达到几十GHz。

巡测深度应该能达到mk量级。

巡测完一个波段需要多长时间: 可以根据大概的技术参数计算一下。大带宽的话,观测一条谱线和观测一个波段用的时间一样。

1平方度的大视场,估计一年就能完成。现在最要紧的是科学目标。
}

\from{Jinhua He <jinhuahe@ynao.ac.cn>}{Jun 11, 2018, 9:51 AM}
\said{
我觉得我们应该再进一步追问:我们应该巡测具体哪几个有机分子?每个有机分子谱线都挺多的,我们应该巡测哪几条跃迁?望远镜能够同时覆盖多宽的带宽?我们应该巡测多大的天区(我对胜利说的“无偏”显然还没有完全理解:)?巡测深度应该是多少?巡测完一个波段需要多长时间?总共需要几年完成?所有答案都需要给出令人信服的理由。我感觉如果我们能够做到这个程度,这一项应该差不多明白了,也可以提技术要求了。

非常近的天体适合做详尽的观测,也许可以考虑将它们从较远的天体里面分离出来,专门列为一个主题。近距离的天体可以包括:太阳系天体、最近的原行星盘(比如140pc的HL Tau和TW Hya等)、最近的残留盘(Debris Disks,比如10pc处的AU Mic等)、最近的过渡盘(Transitional Disks,比如120pc的SR21等)。这些天体的特点是与行星形成直接相关,我们可以专注于研究星际有机物是否以及如何进入类地行星成为生命起源的种子的。

除去这些天体之外的其它天体环境(比如AGB星、SNe、各种星际云、恒星形成区等),适合研究星际有机物起源演化的大图景,可以单独作为另一个主题。

对于星系有机物观测,也许也可以将星系也区分为近邻星系和遥远星系。我想象这两者的分界点可以是:我们是否能够分辨开它们的星系子结构?对近邻星系,也许可以跟银河系内观测星际有机物起源演化大图景那部分合并,用来辅助验证有机物分布与星系子结构的相关性。这可以弄出几个像样的专题巡天观测。对于空间分辨不开的遥远星系,我不清楚我们可以有什么样的有趣的研究主题可以与生命起源紧密联系起来。而对于高红移星系,如果探测到有机分子,也许可以吹吹生命起源时间可以有多早。但是这些题目与PI项目之间就有点儿界限模糊了。我们现在显然应该仅关注规模大的主题。

这样看来,我们可以从逻辑上划分为如下4个部分:
1)有机物无偏巡天
2)行星系统(包括太阳系)中的星际有机物
3)星际有机物的起源演化图景。
4)其它天体化学课题(大杂烩)。

但是,写得这里,我意识到一个问题:前三方面都是60米的专长吗?我感觉这第2点有点儿玄,因为ALMA,NOEMA,SMA等干涉仪比我们这个60米更适合研究它们。那么我们还要花大力气去展开了说吗?如果答案是否定的,那就可以考虑还是把它放回到第3点的起源演化大图景中去吧。

在考虑有机物无偏巡天和起源演化图景这两个大方向的时候,我们也需要格外留心墨西哥的LMT 50m。我们得时时刻刻叨念着我们的独特优势是什么,别浪费大力气去帮LMT设计项目了,呵呵。我们60米独特的技术优势是什么呢?也许是有更高频波段?有更牛X的后端系统?还有吗?

我这些不成熟的想法,提供给各位讨论。欢迎大家都来发言吧。

金华
}


\from{
Jinhua He <jinhuahe@ynao.ac.cn>
}{
Jun 11, 2018, 6:24 AM
}
\said{
谢谢福君分享的提纲,内容很全面。但我在想,我们是否应该突出一下重点?我想象的是重点方面要论证仔细些,篇幅自然要大些,非重点部分压缩些,仅仅是提一提它们以照顾到通用望远镜的全面性而已。

前面大家提到过有机分子巡天是个不错的想法,胜利也在反复提到无偏巡天的概念。我觉得我们值得去重点探讨巡天的可行性,能解决的关键科学问题是什么。也只有这样才可能提得出这个科学目标对技术指标的要求。

对于福君提到的分子在不同天体环境下由简单到复杂的研究层次,我们是不是可以将它们合并成一个大的方面,并且突出其中与生命起源这个核心主题有关的方面?

最后预留一点儿不多的空间,我们就可以浮光掠影地说点儿废话:其实各个方向的天体化学观测课题都是可以做的。

这样的话,我们这部分就有以上三部分内容。福君的提纲就需要做些调整。我总觉得一个重点突出、阐明了重大科学意义的项目建议书要比大而全的好得多。

大家觉得如何呢?

金华}


\from{Fujun Du <fjdu@pmo.ac.cn>
}{Jun 8, 2018, 9:03 PM}

\said{
基于大家的讨论,接下来我们的任务就是为项目建议书写其中的一章。

这一章的书写可以有下面两条线索:
分子:简单 $\rightarrow$ 复杂 $\rightarrow$ 生命前 $\rightarrow$ 生命
天体:太阳系内天体 $\rightarrow$ 星际介质与恒星形成区 $\rightarrow$ 原行星盘 $\rightarrow$ 恒星后期演化 $\rightarrow$ 近邻星系 $\rightarrow$ 高红移宇宙

我的初步设想,是大致以天体的尺度为顺序,讲各类天体中分子物质乃至生命前和生命物质的探测,每部分描述现状和未来60米望远镜的可能贡献。

标题:宇宙有机物分布与生命起源
[标题可能还需要讨论一下,比如如果希望把以磷为主的分子也包含进来的话。]

1. 太阳系内天体的物质组成
1.1 行星和行星大气的物质组成
1.2 小行星的组成
1.3 彗星的组成
1.4 陨石与陨石中的生命物质
[这部分可能不会是60米望远镜未来的重点观测对象,但可以帮助引出在天文环境下搜寻生命相关分子的重要性。]

2. 恒星形成区和星际介质
2.1 冷暗分子云化学
2.2 热分子云核化学
2.3 原行星盘化学
2.4 行星状星云
2.5 超新星遗迹
[这些天体有可能会是重点观测对象]

3. 星系的物质组成
3.1 近邻星系的分子构成
3.2 高红移星系

4. 探索生命起源
4.1 生命前分子和生命分子的探测
4.2 生命分子的手征性起源
4.3 星际尘埃的性质,是否与生命物质有关

5. 分子物质的探测对其它天体物理分支的用途
5.1 对各种基本物理参数的限制
5.2 帮助研究基本天体物理问题
[比如前几天发表的通过CO同位素比限制初始质量函数的工作]

6. 与其它科学领域的关系
6.1 实验分子物理、分子动力学
6.2 分子频谱学
6.3 生物化学
6.4 地球化学
6.5 太阳系天体化学
6.6 大数据、机器学习

杜福君
}


\from{
Jinhua He <jinhuahe@ynao.ac.cn>
}{
Jun 8, 2018, 3:12 AM
}

\said{
LMT的白皮书写得很简略,有些方面的说法可能都不一定正确(比如他们将一个盘放在3pc的距离处,给人错误印象好些LMT真的能分辨出盘结构似的,但我们的60米都未必做得到)。他们也没有太突出LMT特色科学目标。ngVLA的白皮书写得具体深入得多,很多文档都强调了它的独特科学课题。

我们第一步应该写到什么程度,这可能取决于项目组想要什么资料,写给谁看(百姓?领导?国内同行?国际同行?),拿来干什么(宣传?游说?项目评审?)。也许第一步我们可以像LMT那样写得很浅很科普,但是要保证浅而有重点,且无错漏。我们也要思考是不是真像LMT白皮书那样把望远镜能做的事都大概说一遍?还是要突出重点课题,而非重点课题则一笔带过?

感觉LMT就是个通用望远镜项目(我们国内之前做的望远镜项目大都是这个路子,不需要太强的科学目标,呵呵)。但是我们的60米是不是最好要有几个主打的科学特色,就像ngVLA一样?可以想象,将来60米建成之后,我们应该预留一半时间做大规模巡天,实现特色科学目标,另一半时间留给PI项目。而我们现在要写的就应该仅仅是针对特色科学巡天设计目标。没有特色的课题就不用我们太操心了,那是将来PI项目用户们的事情。

但是在LMT白皮书中,我看到了下面这个比较图,用来显示LMT的技术优势空间。这也是我前些时候提出来希望看到的类似比较,明确指出60米的技术优势空间。(但LMT的图中没有说明他们的比较参数是如何定义的,具体是什么含义,参数的计算有何依据。)

\includegraphics[width=0.7\linewidth]{miliigpekkopjoij.png}
}

\from{
Sheng-Li Qin <qin@ynu.edu.cn>
}{
Jun 7, 2018, 1:52 PM
}

\said{
     上次是概括这个子课题要做什么,下一步应该是形成一个书面的白皮书,说服管理者立项。 需要包含的内容可以是:

       1. 意义,

    作这件事情的意义,为什么要作。因为几乎所有人关心这件事情。

    2. 现状,

      国内外研究现状,现有的望远镜不能做好,这个望远镜功能强大,可以做好。

    3. 目标,

      达到什么目标,普查有机物宇宙分布,及其物理化学条件,最终是希望找到生命分子。

4. 观测手段,

  大视场、高分辨率、高灵敏度无偏巡天。

5. 数据处理和诠释,等

   导出物理参数,结合化学模型,探讨有机分子形成,生命前分子以及生命分子产生的物理化学条件。


我觉得大体上应该是这些。


胜利
}

\from{Fujun Du <fjdu@pmo.ac.cn>}
{Jun 7, 2018, 11:22 AM}

\said{
我们可以参考LMT和ngVLA的网站上列的东西,比如:
\url{http://www.lmtgtm.org/the-lmt-book/}
\url{http://ngvla.nrao.edu/page/memos}
或者更具体的,像下面这个:
\url{http://ngvla.nrao.edu/system/media_files/binaries/61/original/McGuire_ExoWP.pdf?1522153180}

细致的研究当然是需要的,只是往往不是短期内能做完的 (比如上次我的邮件里列的部分题目,有的有点大,可能需要更多资金和人员支持)。望远镜的大部分技术参数已经比较清楚了 (除了偏振那部分之外)。
}

\from{Jinhua He <jinhuahe@ynao.ac.cn>}
{Jun 5, 2018, 11:06 AM}
\said{
一平方度确实是很大的,非常不错。我们可以简单利用波束大小和波束的个数估算一下巡天的灵敏度和速度之间的关系,看看是不是真的很容易4PI全天巡测,或者还是主要巡测银道面附近。高银纬天区也许可以根据Planck数据选择有明显辐射的区域局部观测或许也够了吧?但是估算巡天速度需要知道望远镜系统典型的系统温度,建议让技术团队给每个波段估算一个大概数值。

对太阳系小天体的观测也许真是个有趣的方向,不一定等彗星来,大行星,小行星,或者像`Oumuamua那样的偶然天外来客,全都可以是60米的菜。据说`Oumuamua上面红色物质以及无彗星蒸发行为可能就是存在有机物的缘故。:)

对于星系,探测到有机物虽然有趣,但是对生命起源话题会有多大贡献,我还暂时看不出来。银河系里如果遍布有机物,其它星系里面也看到也就不足为奇了。感觉还是太阳附近的天体最切题,银盘内稍远的天体次之,河外星系再次(可以留给那些弄星系的人)。当然,也可以考虑在银道面巡天基础上添加一个Gould Belt巡天来涵盖太阳系附近最活跃的恒星形成区。

金华
}

\from{Sheng-Li Qin <qin@ynu.edu.cn>}
{Jun 5, 2018, 8:41 AM}
\said{ 因为望远镜视场比较大,一次观测可以覆盖一个平方度,效率很高,可以很快能作完整个银河系,这是无偏巡天,能包含所有天体信息。带宽大,能同时观测多个分子。这样的话,我们需要考虑选择那些有机分子。目前一些生命前分子也探测到,这些分子肯定要包含进来,生命分子也要包含进来。以前的观测,仅对几个特殊源观测的详细,无偏巡天能彻底弄清楚有机分子的物理和化学环境,分子特征,能找到生命分子的几率很大,反过来也能弄清楚怎样用不同的分子解决不同的天体物理问题。

而且也要对彗星,近邻星系,红移5以下的天体的天体作观测。

因此题目是宇宙有机物分布及生命起源。
          
所有参会人员都认为,这个子项目未来能作的很好,会带来重大突破。如果能探测到生命分子,就意味着找到了地外生命,可以为生命起源提供线索,当然地外生命不等同于地外文明。

祝好!

胜利
}


\from{Fujun Du <fjdu@pmo.ac.cn>}
{Jun 4, 2018, 11:21 PM}
\said{
基于自己的兴趣和以往经验,以下是我想到的一些可以分头做或者合作做的事情,跟金华列的有部分重叠:
\begin{enumerate}
  \item 给出有可能被探测到的复杂有机分子、生命前分子、生命分子列表,描述探测到它们的意义,通过模拟计算或者外推 ("educated guess") 给出预期出现的物理环境和丰度,以此为依据给出对望远镜设备的定量需求。
  \item 给出预期可能探测到的手性分子列表和丰度;研究手征不对称性的探测方法,以此为依据给出对望远镜设备的定量需求。我相信很多人对怎么探测手征不对称性不是很清楚 (包括最基本的问题:有没有可能探测到?),所以值得弄清楚。
  \item 设计从观测数据自动识别分子谱线的算法并编制程序;这是为了应对今后的大规模谱线巡天数据带来的数据处理压力。
  \item 研究有机分子和生命分子从星际介质到宜居行星的演化路径,为解释观测数据提供理论框架。
\end{enumerate}  
以上每个方向在我看来做起来都不容易,不过如果有结果应该都可以产生不错的研究论文/研究报告/软件程序。

杜福君
}

\from{Chang, Qiang <changqiang@xao.ac.cn>}{Jun 4, 2018, 10:41 PM}
\said{
我看大家一直讨论观测方面的内容,但是我对观测不熟悉,所以没有发言。我就我们最近的模型结果谈一下。
我们刚提交一篇文章到ApJ。根据我们的计算结果,在比较热的源 hot core, hot corino等观测的复杂有机分子
的丰度直接和这些源在冷的阶段,也就是dark cloud的辐射相关。所以观测这些复杂有机分子,
可以追溯过去冷云核阶段的物理条件。

如果有需要,我们这边随身可以作数值计算工作。

常强
}

\from{Jinhua He <jinhuahe@ynao.ac.cn>}{Mon, Jun 4, 10:30 PM}
\said{
谢谢福君分享最新的版本。为了提高通信效率,以后咱们客套的称谓就不要了哈。

我先扔点儿引火柴,希望大家多发言:
1)我个人认为你们提出的针对有机分子的大规模巡天是个好主意。但是巡天的目标分子和跃迁是什么?这个可能需要谨慎选择。去什么地方巡测?银道面?选定的重点区域?多大巡视面积?理由是什么?这些可能需要我们费些周章才能令人信服地搞定。但我感觉这是个值得投入时间精力的活儿。

2)冲击生命分子的直接探测是显而易见的主题。但是如果要把这个科学目标做实,其实也不容易。我不清楚各位在这方面都各自做到什么深度了,反正我自己得坦白地说是什么都还没做,呵呵。(但如果需要的话,我可以随时开始。)我觉得我们可以像暗物质探测项目那样入手,首先努力去说明以60米的预期的观测能力,我们可以将生命分子的探测深度刷新到什么程度。至于可以探测到它们的成功几率,那是个更为复杂的问题。即便我们接受LTE加光学薄条件下的辐射估算结果的话,也有一个它们的气态丰度到底可能有多高的问题。天体化学模拟是一个回答这个问题的渠道之一。在这方面,相信福君、常强和我这边都可以做些数值模拟估算的。(其他各位还有在做天体化学数值模拟的吗?我建议我们整合模拟研究队伍,分享经验,分工合作。)

(可惜常强老师这段时间一直没有回信,不知道新疆是不是又信息封锁了?建议哪位给他打个电话问问情况。)

3)对理化环境和物质循环问题,不知道是否适合合并成一个大类?对具有实践意义的化学变化的讨论总是离不开环境条件的。这方面内容比较丰富,我们也许可以从对环境条件的分类和每个有机分子的化学反应通道情况的评述开始。我这里说的环境条件是指比较具体的小尺度环境;大尺度环境题主我建议放到前面的大规模巡天项目中去考虑。同意否?

4)最后,我觉得原行星盘也可以是我们考虑的一个重点对象。对生命起源主题的理论意义我就不啰嗦了。这里举个实例。邻近恒星形成区中的星周盘的尺度可以被60米在最短的波段里勉强跟环境成分区别开(比如最著名的HL Tau的毫米波连续谱辐射和HCO+ 1-0谱线盘角大小约为1.5~2.0角秒,在140pc的距离处,这相当于210~280AU的线尺度;如果是考虑其它分子,或者考虑更加早期的原行星盘,这个线尺度也可以大得多)。如果60米的波束大小跟盘大小相当,我们就应该可以很好地做专门针对盘的频谱巡天观测,讨论盘整体的化学状态了。当然了,这个不全是60米的技术优势空间,因为ALMA更厉害。但60米的优势是可以在没有干涉仪的图像模糊畸变效应的前提下,可以宽波段谱线巡测,得到更真实可靠的频谱数据。这方面60米比ALMA强,其本质就是比ALMA的4个12米口径的Total Power小弟弟们强,而且可以很专注地做最为完整的观测,建立具有legacy价值的数据库(ALMA分心的事情可是多得很的,今年暴涨到1800多个PI项目申请量就是明证)。

祝好!
金华
}


\from{
Sheng-Li Qin <qin@ynu.edu.cn>
}{
Jun 2, 2018, 2:35 PM
}
\said{
是的,它不是传统意义上的糖分子,只有丙糖才算得上,但乙醇醛进一步反应会生成核糖。在天文上和化学上都把它称作最简单糖分子,是生命前分子,不是生命分子。2017年发表的论文,以及最近前几年都仍称作最简单糖分子。

秦胜利
}

\from{
Li Juan <lijuan@shao.ac.cn>
}{
Jun 2, 2018, 2:21 PM
}
\said{
有个小问题,乙醇醛不能称之为糖分子,Hollis et al. (2000, 2004)文章中的说法是不准确的,具有3个碳原子的单糖,也就是丙糖CH2OH-CHOH-CHO(glyceraldehyde)才可以称之为糖分子。丙糖目前尚未在星际空间探测到。
}



\from{
Jinhua He <jinhuahe@ynao.ac.cn>
}{
Jun 2, 2018, 11:26 AM
}
\said{
大家好!

不好意思,我前两天有点儿忙。今天抽点儿时间来讨论一下前面提到的部分话题。不知道周四周五的现场会议情况怎样?哪位老师能分享一下?

前面我们提到了同位素化学的问题。我希望澄清一下:我说的主要是有机分子的同位素,而不是CO的。由于同位素分子之间具有较为相似的分子质量和能级结构,在涉及到低温化学过程的时候,可能会成为很便利的探针。因此,将同位素分馏效应与有机分子化学结合起来,也许会催生出一些有趣的课题。这方面如果要有所进展,我们可能最好专门安排人去调研总结同位素化学和有机分子化学的概况,然后尝试将它们联系起来,并且找出60米最适合解决的关键问题。

另一个问题是,我觉得我们不应该太担心60米望远镜探测不到丰度低一些的元素的分子的问题,因为这个望远镜的优势就是口径大,灵敏度高。比如张泳建议的含P分子,以及前面提及的同位素分子等,观测它们更能体现60米望远镜优势,我们应该放心地把它们考虑进来。

不过,我们倒是应该讨论如何将这些分子或者元素与生命起源问题联系起来。虽然秦老师已经给出了一个地球生命起源的很流行的看法,但是我们这个项目的目标显然应该是要给出一个更为深刻的见解才好。我的建议有两个:

1)按照张泳提到的说法,需要去调研有机分子的起源和演化路径的研究现状(这又涉及到不同的天体环境,不容易);

2)前面我们提到过60米的一个优势是角分辨率比较高,因此我们可以预期将会对典型的临近天体(比如云核,AGB星周包层等)得到保真度比干涉仪更高的空间图像。我觉得我们需要去调研一下有空间分辨的天体化学(特别是有机分子)观测的研究现状,总结出60米可以重点攻克的关键问题。

扎实的调研应该是个很有好处的步骤。

周末愉快!

金华
}

\from{
Jinhua He <jinhuahe@ynao.ac.cn>
}{
May 31, 2018, 7:40 AM
}
\said{
我不清楚单天线真正的偏振观测如何做,但是ALMA的偏振观测据说过程复杂得多,远远不只是通常谱线观测的两个垂直线偏。
}

\from{
Sheng-Li Qin <qin@ynu.edu.cn>
}{
May 31, 2018, 7:12 AM
}
\said{

我曾处理过herschel 1千多GHz 带宽的数据并作分子证认, 那个数据有两个偏振, 最后加起来提高信噪比。两个偏振的强度几乎一样, 总觉得区分手征分子比较难。

胜利
}


\from{
Jinhua He <jinhuahe@ynao.ac.cn>
}{
May 31, 2018, 4:47 AM
}
\said{

大家好!

我对分子手征性这个话题不太熟悉。今天特地学习了一下。下面有一点点总结分享给各位:

分子手征性概念有两个可能比较容易混淆的定义:一个是根据分子的光学偏振性质(d-,l-),一个是根据几何结构性质来定义(D-,L-)。地球上生物氨基酸普遍左旋(L-)性质指的是后一种定义,也就是它们在几何结构上与具有左旋光学性质(l-)的甘油醛glyceraldehyde的几何结构类似,就定义为了L-氨基酸。因此L-氨基酸不一定具有l-光学偏振性质。这个网页解释得比较清楚:\url{https://en.wikipedia.org/wiki/Absolute_configuration}

我不清楚这与天文文献中提到的圆偏振UV光子选择性离解某一类手征性分子的化学机制是否会有某种内在联系。如果有联系的话,那很可能就意味着空间氨基酸分子的手征性(D-,L-)比例与UV光子圆偏振方向可能并不总是一致的。也就是说,同样右旋偏振的UV光可能破坏一种氨基酸的D-型,而破坏另一种氨基酸的L-型。上面网页中就提到9个地球上的生物氨基酸(L-型)实际上具有d-型偏振性质。这一点我们将来可能需要留意。

我觉得空间有机分子手征性偏离平衡有可能给地球生物氨基酸手征性的偏好提供一个不平衡的初始条件,在后随的演化竞争中产生出目前看到的L-氨基酸主导生命过程的局面。因此,如果60米能够通过偏振来测量各种空间有机分子的这个偏离平衡的状况的话,还是很有意义的。

我也不清楚McGuire说的高信噪比的偏振测量如何能够确定有机分子的手征性。如果是利用它们的起偏性质的话,无论是线偏还是圆偏,可能还需要与其它可能的偏振过程加以区分。粗糙想来确实不易,但也不是不可能。:)
}

\from{
Fujun Du <fjdu@pmo.ac.cn>
}{
May 31, 2018, 1:03 AM
}
\said{
秦老师,张老师,

关于通过观测区分左旋还是右旋,仅凭现在的谱线观测确实无法区分,因为它们的能级都是一样的。

不过我们这里谈论的是未来的仪器,而根据McGuire2016 (\url{http://www.sciencemag.org/cgi/content/full/science.aae0328/DC1}) 文章里的讨论,通过高精度偏振测量应该是有可能测出来的,虽然现有的仪器还做不到;难度会很大 (可能不低于做宇宙学的人探测原初引力波B模的难度),但应该不是不可能。不过这方面的技术和理论我不太熟悉,理解不一定正确。

对手性起源的研究也是ngVLA的关键科学目标之一 (参见\url{http://ngvla.nrao.edu/page/science})。

如果能证明天文环境下会选择性地破坏其中一个,那应该算是个大发现了。

杜福君
}

\from{
Sheng-Li Qin <qin@ynu.edu.cn>
}
{
May 30, 2018, 9:19 PM
}
\said{
赞同,目前的观测无法区分左旋换是右旋。而且有理论认为在天体环境中uv环偏振总是会选择性地破坏其中一个。因此需要放弃
观测区分手征。

含磷分子仅在几个天体探测到,而且很弱。60米的高灵敏度可以做的更好。因为有机分子更丰富,更适合作为主要科学目标。

是的,最初我提出过不同演化阶段、不同光度的源大样本观测。因为观测的谱受激发,而且受演化的影响,大样本观测也许能区分出不同效应的影响。

由于有机分子、生命前分子、生命分子不能合成于早期地球,目前地球上的大部分分子来源于早期恒星形成,通过彗星、陨石带到地球。那么比较彗星和地球以及星际云的化学成分和丰度是一个途径。目前在彗星上探测到了大概一半的星际分子。当然个人觉得生命起源是很小的一部分,是为了说服外边人赞同这个项目。最重要的仍然是星际介质和天体化学研究,弄清楚不同物理环境如何影响分子合成,以及怎样用不同的分子解决不同的天体物理问题这些最基本的问题。

目前已在星际空间探测到氨基酸最直接的前驱,以及合成核糖核酸DNA和脱氧核糖核酸RNA的最相关分子咪唑,由于以前的望远镜探测灵敏度极低,做的不好。而且这些分子不一定在稠密云环境,干涉仪也许不适合,需大样本观测高灵敏度单天线观测。

仅供参考。

秦胜利
}

\from{
张泳 <zhangyong5@mail.sysu.edu.cn>
}{
May 30, 2018, 8:40 PM
}
\said{
手征性分子很有意思,可以检验生物的手征性分子不对称,如果多探测到一个新的就将是很重要的发现,
但是左旋还是右旋没有办法通过射电谱来确定,因此无法测量相对差异。

生命起源课题也不一定局限于复杂分子,比如含磷(组成DNA基本元素)分子目前的探测也不够。我的
拙见是研究天体化学和生命起源,在各个波段都很重要,对于亚毫米波的分子进行大样本(恒星晚期$\rightarrow$
星际 $\rightarrow$ 恒星形成区各个阶段等)高灵敏度观测可以更完整勾画星系物质化学演化图像,可以研究化学
与环境的关系,有可能外推到生命复杂分子形成,另外既可能在强源里发现新分子,也可能首次在弱源中
发现已知分子。

张泳
}

\from{
Jinhua He <jinhuahe@ynao.ac.cn>
}{
May 29, 2018, 10:22 PM
}
\said{
我这里就前面我们的讨论邮件再做个简单讨论。

之前李娟提到了很复杂的分子即便在红外去观测吸收带也不一定容易。也许这样说是对的。但是在毫米波段探测生命的分子的发射线的可行性则要等杜老师说的理论和模拟估算工作给我们提供更深刻一些的认识。把这个亚毫米波60米镜扩展到厘米波段可以考虑,但显然不是目前这个立项阶段的重点。我们目前多集中力量突击重点方向为好,因为我们人力很有限。

IRAS 16293-2422是个不错的hot corino吧?我觉得可以作为尝试模拟估算的天体对象。杜老师准备做它吗?

另外,其它类型的天体是不是也值得考虑?比如大质量恒星的热云核,暗云,弥散云,HII区的PDR,AGB星和SNe等。将这些天体也纳入我们的视野,才可能得到一个较为完整的生命分子及其原材料如何产生、演化并最终进入行星系统的物理图景。

化学和辐射转移模拟可以给出丰度信息和可能的辐射强度。但其中的难点是:1)化学丰度不确定性比较大(也许还是可以在量级上给予模拟结果些许信任);2)复杂分子的能级系统及其跃迁概率系数不一定有现成的数据。秦老师前面说SKA要探测生命大分子需要2000小时积分,不知道这个是如何算出来的?也许那边会有些经验和数据可以分享吧?我们也值得与SKA的估算结果进行比较。

cosmochemistry这个名词,根据wikipedia的解释内容,似乎更多的指行与星系统密切相关的天体化学问题,也包括元素丰度问题。光学波段观测者还将恒星元素丰度问题也称为化学,与我们这里关注的分子化学既有联系又有区别。cosmochemistry这个名词的来历我估计与已经过时的古代宇宙观有关系,可能那时候人们眼里的宇宙cosmo没有超出太阳系多远。我们是不是可以把cosmochemistry意译成行星系统化学(Planetary System Chemistry)或者太阳系化学(Solar system chemistry)或者别的?

行星系统的化学,以及地球化学,确实也是一个值得我们关注的相关领域。要不我们也去这些领域找几个专家来加入,形成跨学科合作团队?
}

\from{
Li Juan <lijuan@shao.ac.cn>
}{
May 28, 2018, 10:29 AM
}
\said{
何老师,您好,
         之所以提出保留几十GHz的观测能力,主要是从天线运行的角度考虑,如果天气状况不允许进行3毫米波段以上的观测,望远镜不至于空等天气变好(不过当时杨台提出了这样的天气应该不多),另一方面几十GHz对于大分子的观测确实有优势。我们之前65米望远镜的工作表明,很多大分子的在银心的分布非常延展,适合于单天线望远镜的观测。个人意见,仅供参考。

祝好,
李娟
}

\from{
Li Juan <lijuan@shao.ac.cn>
}{
May 28, 2018, 5:15 AM
}
\said{
各位老师好,何老师提到的大分子不一定是气态,可能以冰的形式存在,建议在红外波段观测,我之前也跟李菂老师探讨过。他说红外谱线的证认更为困难,那些PAH的发射到现在还没有能够证认出来是具体来自什么分子。

关于谱线的blending,对银心分子云的观测影响比较大(线宽通常在几十km/s),对于类太阳的年轻原恒星,如在复杂有机分子方面的研究热点的IRAS16293,由于谱线线宽要窄的多,影响应该会小很多。spitzer望远镜通过对近邻嵌埋星团的观测,证认出来了星团中的class 0,class I,class II、class III天体,也许我们可以考虑针对年轻的原恒星包括clasd 0和class I做一个人口普查式的工作,看看近邻多少原恒星里面能探测到大分子的发射。

祝好,
李娟
}

\from{
Jinhua He <jinhuahe@ynao.ac.cn>
}{
May 27, 2018, 3:59 AM
}
\said{
正如胜利所说,亚毫米观测太复杂的有机分子是困难的,说60米镜要观测它们容易被人质疑。我建议我们还是以简单一些的有机分子为研究重点,说明它们与生命大分子之间的紧密关系,自然就与生命起源联系起来,这样也可以很有看点儿。生命大分子是否直接来自空间这个问题本身都是有争议的。即便存在,很可能以冰物质的形式存在,而不一定是气态,也许红外吸收带观测是更好的方式。

已经观测到的星际星周分子在CDMS就可以看到。也可以在此基础上基于化学模型提出一些有希望观测到的新分子,但显然也只可能是最简单的类别。这个就可以作为我们说事儿的分子基础了。这方面常强老师在理论上很擅长,能否请常老师做些贡献?
}

\from{
Sheng-Li Qin <slqin@bao.ac.cn>
}{
May 26, 2018, 5:11 PM
}
\said{
是的,干涉仪的short spacing导致两个问题:1, 过滤掉延展的低密度成份,2,导致谱线流量强度有点低。但这对于致密分子探针没有影响。
目前,面临的另一个问题是,观测是二维的,我们所观测到的谱线来自于真实三维空间的所有成份的贡献,而我们基于有限的数据得到的信息,是投影到二维空间上的。 只有多波段、尽可能多的分子谱线协同观测,才能反演得到真实的信息。

目前,所有在用和计划在建的望远镜,都把生命起源列为重要目标,但是也几乎是不可能达到的。这样作的目的,是为了说服官方投资立项。 所以,我们这个项目立足于分子天体化学和有机分子最基本的问题,也要提一下生命分子的探测,比如糖分子以及其它一些生命前分子。

具体是选取一些关键分子,作大样本观测:不同演化序列的源,不同光度的源作系统观测,为化学模型输入提供准确参量。当然,正如金华所说,选取近距离的源。

我们目前需要思考的是,观测那些分子,解决那些问题。

祝好!
胜利}

\from{
Jinhua He <jinhuahe@ynao.ac.cn>
}{
May 26, 2018, 3:51 PM
}
\said{
就目前看来,我和秦老师似乎有了一个共识点:重视简单有机分子的空间分布的系统观测研究。

另外,我觉得干涉仪的missing flux对探测分子谱线一般来讲并不是什么弊病,它漏掉的是延展的空间结构成分而已,只对具有延展角分布的分子有影响。灵敏度的关键还是要看总的望远镜接收面积吧。所以我不建议过分强调这个差异,除非我们真的特别关注邻近天体的延展分布的某些特定分子。

祝好!
金华
}

\from{
Sheng-Li Qin <slqin@bao.ac.cn>
}{
May 26, 2018, 11:11 AM
}
\said{
典型的230 GHz应该能达到5角秒。

能达到mk量级。

肯定要作偏振,这是这个望远镜的主要目标之一,而且高频的偏振强。

对氨基酸分子探测,用这个望远镜几乎是不可能的,因为高频的line blending非常强,而氨基酸分子丰度低,全被淹没在其它信号之下。即使未来的SKA,要想探测氨基酸,需要积分时间2000小时。

探测复杂有机分子,毫米/亚毫米观测是理想的波段,因为其分子特点,大部分分子会在这个波段探测到,当然会有严重的line blending,目前的处理技术可以解决line blending问题。

福君也提到测量星际分子的手征不对称性,需明确手征不对称性的重要性是什么。

在我看来这个望远镜的优势是高的灵敏度,能探测到弱的分子,而且没有干涉仪的空间流量丢失,这样的话,能探测到更多的分子,这是以前的望远镜不可实现,希冀也能探测到新分子。

目前的观测到谱线,导出的物理参数,受到演化、加热机制等的影响,系统观测可以区分这些。

最重要的是,这个领域存在那些问题,通过观测我们要解决什么问题。

胜利
}

\from{
Jinhua He <jinhuahe@ynao.ac.cn>
}{
Sat, May 26, 4:26 AM
}
\said{
我觉得首先要明确一下这个望远镜的大概的技术指标范围,比如主要工作在亚毫米波(什么频率范围?100GHz-1THz吗?),空间分辨率高(1-10角秒?),灵敏度高(望远镜接收面积~60米$^2$  = 12米$^2$ * 25 = 半个ALMA 12-m主阵接收面积),成图观测快(连续谱和谱线计划做多少个波束的接收机阵列?),可以观测偏振(线偏振?)。这些信息可以在考虑科学目时提供一个大概的限制。这些参数显然是各个研究方向都要考虑的,因此建议由总体团队确定出最合适的参数范围,分享给各个学科方向参考。免得大家都去重复思考,而且还可避免出现不同小组估计的参数范围不一致的情况。你们觉得如何?

对于生命起源,我个人感觉探测那些复杂分子在亚毫米波可能比较困难,至少难作为一个主要的科学目标。也许系统地去研究星际有机分子的产生、演化和转移过程比较适合这个单天线望远镜。不知道各位老师有什么不同看法?

对于一般天体化学观测的话题,我暂时没有想到有什么可以称为holy grail的。:)我只能重申我在远程发言中曾提到的:尘埃化学可以有非常丰富的研究内容。但是需要仔细地理一理什么是最有新意的话题。

抛砖引玉吧。希望各位老师多多发言。

祝好!

金华
}

\from{
Fujun Du <fjdu@pmo.ac.cn>
}{
Sat, May 26, 1:24 AM
}
\said{
目前需要做的事情主要还是厘清下述方面:
1. 这个领域的重要科学目标
2. 60米级亚毫米波望远镜能为这些科学目标做些什么
3. 为了实现这些科学目标,对望远镜性能有哪些要求

在我个人看来,目前我们这个方向的科学目标还比较模糊(不了解的人可能以为就是“找外星人”)。到底应该以什么为目标?“holy grail”是什么?探测星际氨基酸分子?测量星际分子的手征不对称性?我还不是很清楚。

杜福君
}

\section{望远镜的初步技术指标}

\subsection{集光面积}
\begin{equation}
  S = \frac{\pi}{4} (60\,\text{m})^2 = 2827~\text{m}^2,
\end{equation}
与 25 个 ALMA 的 12m 镜相当。

\subsection{工作波段}

{\scriptsize
\noindent\begin{tabular}{l|l|l|l|l|l|l|l}
  & 波段 & 频率范围 & 口面 & 大气 & 视场 & $1.22\frac{\lambda}{D}$ & NEFD \\
  &      & (GHz) & 效率 & 透过 & 直径 ($^\circ$)  & ($''$) & (mJy$\cdot$s$^{1/2}$) \\
\hline
1 & 3mm & 75 -- 118 & 0.82         & 0.96 & 1.35 & $17-11$   & 0.30 \\
2 & 2mm & 120 -- 182 & 0.80        & 0.96 & 1.10 & $10-7$    & 0.25 \\
3 & 1mm L & 185 -- 260 & 0.76      & 0.94 & 0.91 & $7-5$     & 0.28 \\
4 & 1mm U & 240 -- 323 & 0.73      & 0.90 & 0.81 & $5-4$     & 0.34 \\
5 & 850 $\mu$m & 327 -- 373 & 0.68 & 0.80 & 0.71 & $3.8-3.4$ & 0.91 \\
6 & 650 $\mu$m & 388 -- 496 & 0.60 & 0.64 & 0.64 & $3.2-2.5$ & 1.93
\end{tabular}
}\\
基于假定 PWV = {1}mm, 俯仰角 = 50$^\circ$,半波前误差 = 30 $\mu$m RMS。
NEFD: Noise-equivalent flux density;
NEP: noise-equivalent-power.


\subsection{多波束}
\begin{itemize}
  \item 连续谱:像元素可到 $10^3$ 的量级
  \item 谱线观测:可做到\emph{几十}个波束
\end{itemize}

\subsection{偏振观测}
能做。具体技术指标需要根据磁场测量和手征超出 (enantiomeric excess) 测量给出的要求来定。

\subsection{指向}

\subsection{台址特性}

\section{本方向的科学目标}

\subsection{星际有机物的广域分布}
  \begin{itemize}
    \item 普查不同天体物理环境下的物质组成和物质循环
    \item 结合化学模型计算,理解这些分子的化学起源
  \end{itemize}
\subsection{探测生命前物质}
  \begin{itemize}
    \item 生命前物质产生的理化环境
    \item 生命前物质的聚合
    \item 手性分子和手征不对称的探测
    \item 氨基乙腈 (与甘氨酸有关)、乙醇醛 (糖分子)、乙二醇 (与 DNA 和 RNA 有关), \ldots
  \end{itemize}
\subsection{探测生命物质}
  \begin{itemize}
    \item 氨基酸,例如甘氨酸
    \item 核糖核酸 (RNA),脱氧核糖核酸 (DNA)
  \end{itemize}

\section{任务分解}

\begin{enumerate}
  \item 给出有可能被探测到的复杂有机分子、生命前分子、生命分子列表,描述探测到它们的意义,通过模拟计算或者推测 (“educated guess”) 给出预期出现的物理环境和丰度,以此为依据给出对望远镜设备的定量需求。\\
  负责人: \\
  参与人: \\
  时限:
  \item 给出预期可能探测到的手性分子列表和丰度;研究手征不对称性的探测方法,以此为依据给出对望远镜设备的定量需求。我相信很多人对怎么探测手征不对称性不是很清楚 (包括最基本的问题:有没有可能探测到?),所以值得弄清楚。\\
  负责人: \\
  参与人: \\
  时限:
  \item 设计从观测数据自动识别分子谱线的算法并编制程序;这是为了应对今后的大规模谱线巡天数据带来的数据处理压力。\\
  负责人: \\
  参与人: \\
  时限:
  \item 研究有机分子和生命分子从星际介质到宜居行星的演化路径,为解释观测数据提供理论框架。\\
  负责人: \\
  参与人: \\
  时限:
\end{enumerate}

\end{document}
